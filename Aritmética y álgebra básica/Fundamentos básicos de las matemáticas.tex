\documentclass[14pt]{extarticle}
\usepackage{graphicx}
\usepackage{titling}
\usepackage{titlesec}
\renewcommand{\contentsname}{Índice}
\usepackage{float}
\usepackage{amsmath}
\usepackage{amssymb}
\usepackage{tcolorbox}


\pretitle{\begin{center}\Huge\bfseries}
\posttitle{\end{center}}
\title{Fundamentos básicos de las matemáticas}
\author{por Santi Martínez}
\date{\today}

\begin{document}
    \maketitle
    \newpage
    \tableofcontents

    \newpage
    \section{Aritmética y álgebra básica}
        Es la rama de las matemáticas que trabaja con \textbf{números} y las \textbf{operaciones básicas} entre ellos.

        \subsection*{\underline{Aritmética}}

            \subsubsection*{Operaciones fundamentales}
                \begin{center}
                $a + b$, $a - b$, $a \times b$, $\frac{a}{b} \; (b \neq 0)$
                \end{center}

            \subsubsection*{Principales propiedades}
                \begin{itemize}
                    \item \textbf{Conmutativa}. $a × b = b × a$
                    \item \textbf{Asociativa}. $(a + b) + c = a + (b + c)$
                    \item \textbf{Distributiva}. $a × (b + c) = a×b + a×c$
                \end{itemize}

            \subsubsection*{Potenciación y redicación}
                \begin{center}
                $a^m \times a^n = a^{m+n} , \quad (a^m)^n = a^{mn} , \sqrt[n]{a^m} = a^{m/n}$
                \end{center}


        \subsection*{\underline{Álgebra}}

                \subsubsection*{Expresiones algebraicas}
                    \begin{center}
                        $P(x)= a_n x^n + a_{n-1} x^{n-1} + ... + a_1 x + a_0$
                    \end{center}

                \subsubsection*{Ecuaciones}
                    \begin{center}
                        $2x + 5 = 11 \longrightarrow x = 3$
                    \end{center}

                \subsubsection*{Identidades y factorización}
                    \begin{center}
                        $(a+b)^2 = a^2 + 2ab + b^2$
                        \\$(a-b)^2 = a^2 - 2ab + b^2$
                        \\$a^2+ b^2 = (a-b)(a+b)$
                    \end{center}

                \subsubsection*{Sistemas de ecuaciones lineales}
                    \[\begin{cases}
                        a_1 x + b_1 y = c_1 \\
                        a_2 x + b_2 y = c_2
                    \end{cases}
                    \quad \Longrightarrow \quad
                    x = \frac{c_1 b_2 - c_2 b_1}{a_1 b_2 - a_2 b_1}, \;\;
                    y = \frac{a_1 c_2 - a_2 c_1}{a_1 b_2 - a_2 b_1}
                    \]

                \subsubsection*{Funciones}
                    \begin{center}
                        \textbf{Función cuadrática}: $f(x) = ax^2 + bx + c$ \\
                        \textbf{Función lineal}: $f(x) = mx + b$
                    \end{center}



    \newpage
    \section{Ecuaciones y desigualdades}

        \subsection{Ecuaciones}
            Una ecuación es una \textbf{igualdad} que contienen una o más variables. Resolverla significa hallar los valores que completan dicha igualdad.
            \\\\Tipos principales:
                        
            \subsubsection*{Lineales}
                \begin{center}
                    $ax + b+ 0 \longrightarrow x = (-b / a)$
                \end{center}

            \subsubsection*{Cuadráticas}
                \[ax^2 + bx + c = 0 \quad \Longrightarrow \quad
                x = \frac{-b \pm \sqrt{b^2 - 4ac}}{2a}\]

            \subsubsection*{Polinómicas}
                \[P(x) = 0 \quad , \quad
                P(x) \in \mathbb{R} [x]\]

        \newpage
        \subsection{Desigualdades}
            Una desidgualdad \textbf{compara dos expresiones algebraicas} mediante los símbolos:
            \begin{center}
                $<,>,\le,\ge$
            \end{center}

            \subsubsection*{Propiedades básicas}
                Sumar o restar el mismo número no cambia el sentido de la desigualdad:
                    \[a < b \quad \Longrightarrow \quad
                    a + c < b + c\]

                Multiplicar o dividir por un \textbf{número positivo} mantiene el sentido; por un \textbf{número negativo}, invierte el sentido:
                    \[
                    a < b \quad \Longrightarrow \quad
                    -a > -b\]

            \subsubsection*{Ejemplo lineal}
                    \[2x -3 < 5 \quad \Longrightarrow \quad
                    2x < 8 \quad \Longrightarrow \quad
                    x < 4\]

            \subsubsection*{Desigualdad cuadrática}
                    Resolver: $2x^2 + bx + c > 0$
                    \begin{enumerate}
                        \item Hallar raices de $ax^2 + bx + c = 0$.
                        \item Analizar el signo del trinomio en los intervalos determinados por las raices.
                        \item Elegir los intervalos dende la desigualdad se cumple.
                    \end{enumerate}


    \newpage
    \section{Funciones y gráficas}
            Una función es una relación entre dos conjuntos, donde cada elemento del dominio se asocia con un único elemento del codominio.
            \[f : X \longrightarrow Y
            \quad , \quad
            x \mapsto f(x)\]
            \\$\mapsto$: x se le asigna x²

            \subsection*{Funciones lineales}
                \[f(x) = mx + b\]
                Es una recta que sube o baja según $m$ y que cruza el eje $y$ en $b$.

            \subsection*{Funciones cuadráticas}
                \[f(x) = ax^2 + bx + c\]
                Parábola que abre hacia arriba si $a>0$ y hacia abajo si $a<0$.
                \\\\Vértice: $\left(-\frac{b}{2a}, f\left(-\frac{b}{2a}\right)\right)$

            \subsection*{Funciones polinómicas}
                $f(x)= a_n x^n + ... + a_1 x + a_0$
                \\\\Comportamiento determinado por grado $n$ y coeficiente principal $a_n$.

            \subsection*{Funciones racionales}
                \[f(x)= \frac{P(x)}{Q(x)}
                \quad , \quad
                Q(x) \neq 0\]
                \\\\Posibles \texttt{asintotas}* verticales y horizontales
                \\*\textbf{Asíntota}: Línea recta a la que una curva se acerca cada vez más cuando $x$
                o $y$ tienden a un valor extremo(muy grande o muy pequeño), pero sin llegar a tocarla.

            \subsection*{Funciones exponenciales y logarítmicas}
                \[f(x)= a^x
                \quad , \quad
                g(x) = \log_ax\]
                \\Exponenciales siempre positivas; logaritmos definidas para $x > 0$.


            \subsection*{Gráficas de funciones}
                \begin{itemize}
                    \item \textbf{Dominio}: Conjunto de valores posibles de $x$.
                    \item \textbf{Rango}: Conjunto de valores posibles de $f(x)$.
                    \item \textbf{Interceptos}:
                            \\x-intercepto: $f(x) = 0 \Longrightarrow x$
                            \\y-intercepto: $x = 0 \Longrightarrow f(0)$
                    \item \textbf{Crecimiento / Decrecimiento}: Analizar la derivada $f'(x)$ si aplica.
                    \item \textbf{Simetría}: 
                            \\Par: $f(-x) = f(x)$
                            \\Impar: $f(-x) = -f(x)$
                \end{itemize}
            

    \newpage
    \section{Trigonometría}
            Estudia las relaciones entre los ángulos y los lados de un triángulo, y las funciones que describen esas relaciones en el \textbf{círculo unitario}.

            \subsection*{Razones trigonométricas en el triángulo rectángulo}
                Para un triángulo rectángulo con ángulo agudo $\theta$*
                \[\sin \theta = \frac{cateto-opuesto}{hipotenusa}
                \quad , \quad
                \cos \theta = \frac{cateto-adyacente}{hipotenusa}
                \quad , \quad
                \tan \theta = \frac{cateto-opuesto}{cateto-adyacente}\]

                \textbf{$*\theta$}: Relación entre ángulos y lados en un triángulo rectángulo.

                Otras razones derivadas:
                \[\csc \theta = \frac{1}{\sin \theta}
                \quad , \quad
                \sec \theta = \frac{1}{\cos \theta}
                \quad , \quad
                \cot \theta = \frac{1}{\tan \theta}\]

            \subsection*{Identidades trigonométricas fundamentales}

                \begin{center}
                $\sin^2 \theta + \cos^2 \theta = 1$ \\
                $1 + \tan^2 \theta = \sec^2 \theta$ \\
                $1 + \cot^2 \theta = \csc^2 \theta$
                \end{center}

            \subsection*{Fórmulas de adición y sustracción}

                \begin{center}
                $\sin(\alpha \pm \beta) = \sin \alpha \cos \beta \pm \cos \alpha \sin \beta$ \\
                $\cos(\alpha \pm \beta) = \cos \alpha \cos \beta \pm \sin \alpha \sin \beta$ \\
                $\sin(\alpha \pm \beta) = \frac{\tan \alpha \pm \tan \beta}{1 \mp \tan \alpha \tan \beta} $
                \end{center}

            \newpage
            \subsection*{Doble y mitad de ángulo}
                \begin{flushleft}
                \begin{center}
                $\sin(2\theta) = 2 \sin\theta\cos\theta$\\
                $\cos(2\theta) = \cos^2\theta - \sin^2\theta = 2\cos^2\theta - 1 = 1 - 2\sin^2\theta$\\
                $\tan(2\theta) = \frac{2\tan\theta}{1 - \tan^2\theta}$
                \end{center}
                \end{flushleft}


            \subsection*{Funciones trigonométricas en el círculo unitario}
                Relacionan los ángulos con las coordenadas de los puntos en un circulo de radio 1 centrado en el origen (0,0)
                \begin{center}
                \begin{itemize}
                    \item Círculo unitario: radio $r = 1$, centro en $(0,0)$
                    \item Ángulo $\theta$: medido desde el eje positivo $x$, en sentido contrario a las manecillas del reloj
                    \item Coordenadas de un punto: $(x, y) = (\cos\theta, \sin\theta)$
                \end{itemize}
                \end{center}

                Funciones prinicpales:
                \begin{center}
                \begin{itemize}
                    \item Seno: $\sin\theta = y$
                    \item Coseno: $\cos\theta = x$
                    \item Tangente: $\tan\theta = \frac{y}{x}$ (si $x \neq 0$)
                    \item Cotangente: $\cot\theta = \frac{x}{y}$ (si $y \neq 0$)
                    \item Secante: $\sec\theta = \frac{1}{x}$ (si $x \neq 0$)
                    \item Cosecante: $\csc\theta = \frac{1}{y}$ (si $y \neq 0$)
                \end{itemize}
                \end{center}


    \newpage
    \section{Geometría analítica}
        Combina álgebra y geometría. Sirve para estudiar figuras geométricas usando coordenadas y ecuaciones.
        \\Es decir, en lugar de dibujar solo un triángulo o una línea, podemos respesentarlos con números y fórmulas, lo que permite calcular distancias, pendientes, puntos de insercción, etc.

        \begin{itemize}
            \item Se trabaja normalmente en plano cartesiano.
            \item Las figuras como líneas, rectas, círculos y parábolas se pueden describir con ecuaciones algebraicas.
        \end{itemize}

        En resumen, \textbf{traduce la geometría en números y ecuaciones}.


    \newpage
    \section{Límites y continuidad}

        \subsection*{Límites}
            El límite de una función nos dice hacia qué valor se acerca la función cuando la variable $x$ se acerca a un número específico
            \\\\\textbf{Idea clave}: No importa lo que pase en ese punto, sino hacia qué valor se aproxima la función.

            \[\lim_{x \longrightarrow a} f(x) = L\]

            \begin{center}
                "El límite de $f(x)$ cuando $x$ tiende a $a$ es $L$".
            \end{center}

            Ante $\frac{0}{0}$, al ser indeterminada, habría que factorizar.

        \newpage
        \subsection*{Continuidad}
            Una función es continua en un punto si no tiene saltos, agujeros ni interrupciones.

            \subsubsection*{Condiciones para que f(x) sea continua en x=a}
                \begin{enumerate}
                    \item $f(a)$ existe
                    \item $\lim_{x \longrightarrow a} f(x)$ existe
                    \item $\lim_{x \longrightarrow a} f(x) = f(a)$
                \end{enumerate}
                Si alguna de estas falla, la función \textbf{no es continua} en es punto.

                Ejemplo:
                \[f(x) = x^2\]
                \begin{itemize}
                    \item En $x=3$, tenemos $f(3)=9$
                    \item $\lim_{x \longrightarrow 3} x^2 = 9$
                    \item Coinciden $\longrightarrow$ La función es continua en $x=3$
                \end{itemize}


    \newpage
    \section{Derivadas y aplicaciones}
        La derivada de una función mide \textbf{la velocidad de cambio} de esa función en un punto.
        \begin{itemize}
            \item Qué tan rápido cambia $y$ y cuando cambia $x$.
            \item En geometría: es la \textbf{pendiente de la tangente a la curva} en un punto.
        \end{itemize}
        \[f'(x) \quad o \quad \frac{dy}{dx}\]

        \subsection*{Reglas de implementación}
            Regla del potencia: $\frac{d}{dx} (x^n) = nx^{n-1}$\\\\
            Regla del producto: $(uv)' = u'v+uv'$\\\\
            Regla de Suma/Resta: $(u\pm v')=u'\pm v'$\\\\
            Regla de cociente: $\left(\frac{u}{v}\right)'= \frac{u'v-uv'}{v^2}$\\\\
            Cadena: $(f(g(x)))' = f(g(x)) g'(x)$


        \subsection*{Interpretación}
            Si $f(x) = x^2$, en $x=3$
            \[f'(x) = 2 \times 3 =6\]
            Esto significa que \textbf{$x=3$, la curva sube 6 unidades de $y$ por cada unidad que avanzamos de $x$}.
            \\\\Geométricamente: \textbf{la pendiente de la tangente en $x=3$ es 6.}

        \subsection*{Derivadas básicas}
            \begin{itemize}
                \item $(c)' = 0$
                \item $(x)' = 1$
                \item $(e^x)' = e^x$
                \item $(a^x)' = a^x \ln (a)$
                \item $(\ln x)' = 1/x$
                \item $(\sin x)' = \cos x$
                \item $(\cos x) = -\sin x'$
                \item $(\tan x)' = \sec^2 x$
            \end{itemize}

    \newpage
    \section{Integrales}
        La integración es la operación inversa de la derivación y se utiliza para calculas áreas, volúmenes y acumulaciones.
            \subsection*{\underline{Reglas de integración}}
            \begin{itemize}
                \item \textbf{Potencia:} $\int x^n dx = \frac{x^{n+1}}{n+1} + C \quad , \quad n\neq-1$
                \item \textbf{Exponenciales:} $\int\frac{1}{x}dx=\ln {|x|} + c$
                \item \textbf{Logaritmicas:} $\int e^xdx=e^x+c$
                \item \textbf{Suma / resta:} $\int \sin x  dc = -\cos x +c$
                \item \textbf{Reglas trigonométricas:}
                    \[
                    \int \sin x\,dx = -\cos x + C \quad 
                    \int \cos x\,dx = \sin x + C
                    \]
                    \[
                    \int \sec^2 x\,dx = \tan x + C \quad 
                    \int \csc^2 x\,dx = -\cot x + C
                    \]
                    \[
                    \int \sec x \tan x\,dx = \sec x + C \quad 
                    \int \csc x \cot x\,dx = -\csc x + C
                    \]
            \end{itemize}

        \subsection*{\underline{Integral indefinida}}
            Representa una \textbf{familia de funicones} cuya derivada es $f(x)$:
            \[ \int f(x)dx=F(x)+C\]
            donde $C$ es la constante de integración


        \subsection*{\underline{Integral definida}}
            Representa el \textbf{área bajo la curva} de $f(x)$ entre $a$ y $b$
            \[\int_{a}^{b} f(x) dx = F(b) - F(a)\]
            Propiedades:
            \[\int_{a}^{a} f(x)dx=0\]
            \[\int_{a}^{b} f(x)dx = - \int_{b}^{a} f(x)dx\]
            \[\int_{a}^{b} [f(x)+g(x)]dx = \int_{a}^{b} f(x)dx + \int_{a}^{b} g(x)dx\]\\

        \subsection*{\underline{Métodos de integración}}
            \subsubsection*{Por sustitución}
                Si $u = g(x)$, entonces:
                \[\int f(g(x))g'(x)dx = \int f(u)du\]
            \subsubsection*{Por partes}
                \[\int u dv = uv - \int v du\]
            \subsubsection*{Fracciones parciales}
                Descomposición de funciones racionales:
                \[\int \frac{P(x)}{Q(x)}dx\]
            \subsubsection*{Trigonométricas y sustituciones especiales}
                \begin{itemize}
                    \item $x=a \sin \theta \longrightarrow $ para $\sqrt{a^2 - x^2}$
                    \item $x=a \tan \theta \longrightarrow $ para $\sqrt{a^2 + x^2}$
                \end{itemize}

        \subsection*{\underline{Teorema fudamental del cálculo}}
            Conecta derivación e integración:
            \[\frac{d}{dx}   \left(\int_{a}^{x} f(t)dt \right) = f(x)\]
    \newpage
    \section{Matrices y determinantes}


    \newpage
    \section{Vectores, espacios vectoriales, álgebra lineal básica}


    \newpage
    \section{Lógica matemática y combinatoria}





\end{document}